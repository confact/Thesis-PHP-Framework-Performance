% THIS IS SIGPROC-SP.TEX - VERSION 3.1
% WORKS WITH V3.2SP OF ACM_PROC_ARTICLE-SP.CLS
% APRIL 2009
%
% It is an example file showing how to use the 'acm_proc_article-sp.cls' V3.2SP
% LaTeX2e document class file for Conference Proceedings submissions.
% ----------------------------------------------------------------------------------------------------------------
% This .tex file (and associated .cls V3.2SP) *DOES NOT* produce:
%       1) The Permission Statement
%       2) The Conference (location) Info information
%       3) The Copyright Line with ACM data
%       4) Page numbering
% ---------------------------------------------------------------------------------------------------------------
% It is an example which *does* use the .bib file (from which the .bbl file
% is produced).
% REMEMBER HOWEVER: After having produced the .bbl file,
% and prior to final submission,
% you need to 'insert'  your .bbl file into your source .tex file so as to provide
% ONE 'self-contained' source file.
%
% Questions regarding SIGS should be sent to
% Adrienne Griscti ---> griscti@acm.org
%
% Questions/suggestions regarding the guidelines, .tex and .cls files, etc. to
% Gerald Murray ---> murray@hq.acm.org
%
% For tracking purposes - this is V3.1SP - APRIL 2009

\documentclass{acm_proc_article-sp}

\begin{document}

\title{{\ttlit Platform} Performance for Web Development in the Cloud}
\subtitle{Topic Summary}
%
% You need the command \numberofauthors to handle the 'placement
% and alignment' of the authors beneath the title.
%
% For aesthetic reasons, we recommend 'three authors at a time'
% i.e. three 'name/affiliation blocks' be placed beneath the title.
%
% NOTE: You are NOT restricted in how many 'rows' of
% "name/affiliations" may appear. We just ask that you restrict
% the number of 'columns' to three.
%
% Because of the available 'opening page real-estate'
% we ask you to refrain from putting more than six authors
% (two rows with three columns) beneath the article title.
% More than six makes the first-page appear very cluttered indeed.
%
% Use the \alignauthor commands to handle the names
% and affiliations for an 'aesthetic maximum' of six authors.
% Add names, affiliations, addresses for
% the seventh etc. author(s) as the argument for the
% \additionalauthors command.
% These 'additional authors' will be output/set for you
% without further effort on your part as the last section in
% the body of your article BEFORE References or any Appendices.

\numberofauthors{2} %  in this sample file, there are a *total*
% of EIGHT authors. SIX appear on the 'first-page' (for formatting
% reasons) and the remaining two appear in the \additionalauthors section.
%
\author{
% You can go ahead and credit any number of authors here,
% e.g. one 'row of three' or two rows (consisting of one row of three
% and a second row of one, two or three).
%
% The command \alignauthor (no curly braces needed) should
% precede each author name, affiliation/snail-mail address and
% e-mail address. Additionally, tag each line of
% affiliation/address with \affaddr, and tag the
% e-mail address with \email.
%
% 1st. author
\alignauthor
H�kan Nyl�n\titlenote{Nyl�n insisted his name be first.}\\
       \affaddr{Blekinge Institute of Technology}\\
       \affaddr{Valhallav�gen 1}\\
       \affaddr{Karlskrona, Sweden}\\
       \email{hakan@dun.se}
% 2nd. author
\alignauthor
Sundstr�m. Robert\\
       \affaddr{Blekinge Institute of Technology}\\
       \affaddr{Valhallav�gen 1}\\
       \affaddr{Karlskrona, Sweden}\\
       \email{robert.sundstrom@live.se}
}
% There's nothing stopping you putting the seventh, eighth, etc.
% author on the opening page (as the 'third row') but we ask,
% for aesthetic reasons that you place these 'additional authors'
% in the \additional authors block, viz.

\date{7 February, 2012}
% Just remember to make sure that the TOTAL number of authors
% is the number that will appear on the first page PLUS the
% number that will appear in the \additionalauthors section.

\maketitle

% A category with the (minimum) three required fields
\category{H.4}{Information Systems Applications}{Miscellaneous}
%A category including the fourth, optional field follows...
\category{D.2.8}{Software Engineering}{Metrics}[complexity measures, performance measures]

\terms{Theory}

\section{The {\secit Idea} of The Paper}
Typically, the body of a paper is organized
into a hierarchical structure, with numbered or unnumbered
headings for sections, subsections, sub-subsections, and even
smaller sections.  The command \texttt{{\char'134}section} that
precedes this paragraph is part of such a
hierarchy.\footnote{This is the second footnote.  It
starts a series of three footnotes that add nothing
informational, but just give an idea of how footnotes work
and look. It is a wordy one, just so you see
how a longish one plays out.} \LaTeX\ handles the numbering
and placement of these headings for you, when you use
the appropriate heading commands around the titles
of the headings.  If you want a sub-subsection or
smaller part to be unnumbered in your output, simply append an
asterisk to the command name.  Examples of both
numbered and unnumbered headings will appear throughout the
balance of this sample document.

Because the entire article is contained in
the \textbf{document} environment, you can indicate the
start of a new paragraph with a blank line in your
input file; that is why this sentence forms a separate paragraph.

\subsection{Type Changes and {\subsecit Special} Characters}
We have already seen several typeface changes in this sample.  You
can indicate italicized words or phrases in your text with
the command \texttt{{\char'134}textit}; emboldening with the
command \texttt{{\char'134}textbf}
and typewriter-style (for instance, for computer code) with
\texttt{{\char'134}texttt}.  But remember, you do not
have to indicate typestyle changes when such changes are
part of the \textit{structural} elements of your
article; for instance, the heading of this subsection will
be in a sans serif\footnote{A third footnote, here.
Let's make this a rather short one to
see how it looks.} typeface, but that is handled by the
document class file. Take care with the use
of\footnote{A fourth, and last, footnote.}
the curly braces in typeface changes; they mark
the beginning and end of
the text that is to be in the different typeface.

You can use whatever symbols, accented characters, or
non-English characters you need anywhere in your document;
you can find a complete list of what is
available in the \textit{\LaTeX\
User's Guide}\cite{Lamport:LaTeX}.

\subsection{Math Equations}
You may want to display math equations in three distinct styles:
inline, numbered or non-numbered display.  Each of
the three are discussed in the next sections.

\subsubsection{Inline (In-text) Equations}
A formula that appears in the running text is called an
inline or in-text formula.  It is produced by the
\textbf{math} environment, which can be
invoked with the usual \texttt{{\char'134}begin. . .{\char'134}end}
construction or with the short form \texttt{\$. . .\$}. You
can use any of the symbols and structures,
from $\alpha$ to $\omega$, available in
\LaTeX\cite{Lamport:LaTeX}; this section will simply show a
few examples of in-text equations in context. Notice how
this equation: \begin{math}\lim_{n\rightarrow \infty}x=0\end{math},
set here in in-line math style, looks slightly different when
set in display style.  (See next section).

\subsubsection{Display Equations}
A numbered display equation -- one set off by vertical space
from the text and centered horizontally -- is produced
by the \textbf{equation} environment. An unnumbered display
equation is produced by the \textbf{displaymath} environment.

Again, in either environment, you can use any of the symbols
and structures available in \LaTeX; this section will just
give a couple of examples of display equations in context.
First, consider the equation, shown as an inline equation above:
\begin{equation}\lim_{n\rightarrow \infty}x=0\end{equation}
Notice how it is formatted somewhat differently in
the \textbf{displaymath}
environment.

%
% The following two commands are all you need in the
% initial runs of your .tex file to
% produce the bibliography for the citations in your paper.
\bibliographystyle{abbrv}
\bibliography{sigproc}  % sigproc.bib is the name of the Bibliography in this case
% You must have a proper ".bib" file
%  and remember to run:
% latex bibtex latex latex
% to resolve all references
%
% ACM needs 'a single self-contained file'!
%
%APPENDICES are optional
%\balancecolumns

\balancecolumns
% That's all folks!
\end{document}
