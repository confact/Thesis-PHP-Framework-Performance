% THIS IS SIGPROC-SP.TEX - VERSION 3.1
% WORKS WITH V3.2SP OF ACM_PROC_ARTICLE-SP.CLS
% APRIL 2009
%
% It is an example file showing how to use the 'acm_proc_article-sp.cls' V3.2SP
% LaTeX2e document class file for Conference Proceedings submissions.
% ----------------------------------------------------------------------------------------------------------------
% This .tex file (and associated .cls V3.2SP) *DOES NOT* produce:
%       1) The Permission Statement
%       2) The Conference (location) Info information
%       3) The Copyright Line with ACM data
%       4) Page numbering
% ---------------------------------------------------------------------------------------------------------------
% It is an example which *does* use the .bib file (from which the .bbl file
% is produced).
% REMEMBER HOWEVER: After having produced the .bbl file,
% and prior to final submission,
% you need to 'insert'  your .bbl file into your source .tex file so as to provide
% ONE 'self-contained' source file.
%
% Questions regarding SIGS should be sent to
% Adrienne Griscti ---> griscti@acm.org
%
% Questions/suggestions regarding the guidelines, .tex and .cls files, etc. to
% Gerald Murray ---> murray@hq.acm.org
%
% For tracking purposes - this is V3.1SP - APRIL 2009

\documentclass{acm_proc_article-sp}

\begin{document}
\bibliographystyle{unsrt}

\title{{\ttlit Platform} Performance for Web Development}
\subtitle{Topic Summary}
%
% You need the command \numberofauthors to handle the 'placement
% and alignment' of the authors beneath the title.
%
% For aesthetic reasons, we recommend 'three authors at a time'
% i.e. three 'name/affiliation blocks' be placed beneath the title.
%
% NOTE: You are NOT restricted in how many 'rows' of
% "name/affiliations" may appear. We just ask that you restrict
% the number of 'columns' to three.
%
% Because of the available 'opening page real-estate'
% we ask you to refrain from putting more than six authors
% (two rows with three columns) beneath the article title.
% More than six makes the first-page appear very cluttered indeed.
%
% Use the \alignauthor commands to handle the names
% and affiliations for an 'aesthetic maximum' of six authors.
% Add names, affiliations, addresses for
% the seventh etc. author(s) as the argument for the
% \additionalauthors command.
% These 'additional authors' will be output/set for you
% without further effort on your part as the last section in
% the body of your article BEFORE References or any Appendices.

\numberofauthors{2} %  in this sample file, there are a *total*
% of EIGHT authors. SIX appear on the 'first-page' (for formatting
% reasons) and the remaining two appear in the \additionalauthors section.
%
\author{
% You can go ahead and credit any number of authors here,
% e.g. one 'row of three' or two rows (consisting of one row of three
% and a second row of one, two or three).
%
% The command \alignauthor (no curly braces needed) should
% precede each author name, affiliation/snail-mail address and
% e-mail address. Additionally, tag each line of
% affiliation/address with \affaddr, and tag the
% e-mail address with \email.
%
% 1st. author
\alignauthor
Nyl�n. H�kan\\
       \affaddr{Blekinge Institute of Technology}\\
       \affaddr{Valhallav�gen 1}\\
       \affaddr{Karlskrona, Sweden}\\
       \email{hakan@dun.se}
% 2nd. author
\alignauthor
Sundstr�m. Robert\\
       \affaddr{Blekinge Institute of Technology}\\
       \affaddr{Valhallav�gen 1}\\
       \affaddr{Karlskrona, Sweden}\\
       \email{robert.sundstrom@live.se}
}
% There's nothing stopping you putting the seventh, eighth, etc.
% author on the opening page (as the 'third row') but we ask,
% for aesthetic reasons that you place these 'additional authors'
% in the \additional authors block, viz.

\date{7 February, 2012}
% Just remember to make sure that the TOTAL number of authors
% is the number that will appear on the first page PLUS the
% number that will appear in the \additionalauthors section.

\maketitle

% A category with the (minimum) three required fields
%\category{H.4}{Information Systems Applications}{Miscellaneous}
%A category including the fourth, optional field follows...
%\category{D.2.8}{Software Engineering}{Metrics}[complexity measures, performance measures]

%\terms{Theory}

\section{The Background of the {\secit Idea}}
The whole idea started as a discussion about which programming language is better on different tasks. The whole thing grow to a discuss and wonder about which platform is better on having high traffics and which is more trustworthy in performance as a visitor to the site.

We found similar work such as \cite{ieee:javaphp}. These are old, take up techniques that is old and isn't testing .NET, as we will do. We See this as a important thing to test due many people, both academical and in business as a close mind while choosing a programming language.

We will test performance and high traffic on the three biggest programming language for web development. PHP, Java and .NET. This will be tested in Unix except .NET due to requirement of using it on Windows.

Because the entire article is contained in
the \textbf{document} environment, you can indicate the
start of a new paragraph with a blank line in your
input file; that is why this sentence forms a separate paragraph.

\subsection{Focus Areas}
We will focus on performance and will also fake a high traffic to see how different platforms will handle it. This is a important test to see how a platform can be used in different business and in different websites sizes.

performance can be different things. But we will focus mostly on page loads and how cpu and memory is being used. 

This focus areas is important to see how a platform can be used in different business and in different websites sizes.

\subsection{Goals}
How goals is to know which platform is good in different ways. Like high traffic, cpu and memory usage.

%
% The following two commands are all you need in the
% initial runs of your .tex file to
% produce the bibliography for the citations in your paper.
\bibliographystyle{abbrv}
\bibliography{sigproc}  % sigproc.bib is the name of the Bibliography in this case
% You must have a proper ".bib" file
%  and remember to run:
% latex bibtex latex latex
% to resolve all references
%
% ACM needs 'a single self-contained file'!
%
%APPENDICES are optional
%\balancecolumns

\balancecolumns
% That's all folks!
\end{document}
