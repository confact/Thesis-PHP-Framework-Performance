% THIS IS SIGPROC-SP.TEX - VERSION 3.1
% WORKS WITH V3.2SP OF ACM_PROC_ARTICLE-SP.CLS
% APRIL 2009
%
% It is an example file showing how to use the 'acm_proc_article-sp.cls' V3.2SP
% LaTeX2e document class file for Conference Proceedings submissions.
% ----------------------------------------------------------------------------------------------------------------
% This .tex file (and associated .cls V3.2SP) *DOES NOT* produce:
%       1) The Permission Statement
%       2) The Conference (location) Info information
%       3) The Copyright Line with ACM data
%       4) Page numbering
% ---------------------------------------------------------------------------------------------------------------
% It is an example which *does* use the .bib file (from which the .bbl file
% is produced).
% REMEMBER HOWEVER: After having produced the .bbl file,
% and prior to final submission,
% you need to 'insert'  your .bbl file into your source .tex file so as to provide
% ONE 'self-contained' source file.
%
% Questions regarding SIGS should be sent to
% Adrienne Griscti ---> griscti@acm.org
%
% Questions/suggestions regarding the guidelines, .tex and .cls files, etc. to
% Gerald Murray ---> murray@hq.acm.org
%
% For tracking purposes - this is V3.1SP - APRIL 2009

\documentclass{acm_proc_article-sp}

\begin{document}

\title{{\ttlit PHP Framework} Performance for Web Development}
\subtitle{Between PHP Frameworks Codeigniter and CakePHP}
%\subtitle{Topic Summary}
%
% You need the command \numberofauthors to handle the 'placement
% and alignment' of the authors beneath the title.
%
% For aesthetic reasons, we recommend 'three authors at a time'
% i.e. three 'name/affiliation blocks' be placed beneath the title.
%
% NOTE: You are NOT restricted in how many 'rows' of
% "name/affiliations" may appear. We just ask that you restrict
% the number of 'columns' to three.
%
% Because of the available 'opening page real-estate'
% we ask you to refrain from putting more than six authors
% (two rows with three columns) beneath the article title.
% More than six makes the first-page appear very cluttered indeed.
%
% Use the \alignauthor commands to handle the names
% and affiliations for an 'aesthetic maximum' of six authors.
% Add names, affiliations, addresses for
% the seventh etc. author(s) as the argument for the
% \additionalauthors command.
% These 'additional authors' will be output/set for you
% without further effort on your part as the last section in
% the body of your article BEFORE References or any Appendices.

\numberofauthors{1} %  in this sample file, there are a *total*
% of EIGHT authors. SIX appear on the 'first-page' (for formatting
% reasons) and the remaining two appear in the \additionalauthors section.
%
\author{
% You can go ahead and credit any number of authors here,
% e.g. one 'row of three' or two rows (consisting of one row of three
% and a second row of one, two or three).
%
% The command \alignauthor (no curly braces needed) should
% precede each author name, affiliation/snail-mail address and
% e-mail address. Additionally, tag each line of
% affiliation/address with \affaddr, and tag the
% e-mail address with \email.
%
% 1st. author
\alignauthor
Nyl�n. H�kan\\
       \affaddr{Blekinge Institute of Technology}\\
       \affaddr{Valhallav�gen 1}\\
       \affaddr{Karlskrona, Sweden}\\
       \email{hakan@dun.se}
}
% There's nothing stopping you putting the seventh, eighth, etc.
% author on the opening page (as the 'third row') but we ask,
% for aesthetic reasons that you place these 'additional authors'
% in the \additional authors block, viz.

\date{11 March, 2012}
% Just remember to make sure that the TOTAL number of authors
% is the number that will appear on the first page PLUS the
% number that will appear in the \additionalauthors section.

\maketitle

% A category with the (minimum) three required fields
%\category{H.4}{Information Systems Applications}{Miscellaneous}
%A category including the fourth, optional field follows...
%\category{D.2.8}{Software Engineering}{Metrics}[complexity measures, performance measures]

%\terms{Theory}

\section{The Background of the {\secit Idea}}
The frameworks is for the developers help, but is it good for a visitor? I started to think how the performance is between those two and if a visitor can see it as in speed and such. No already done work or article has been done as I could see, only some on codeigniter \cite{MVC:codeigniter} and Cakephp \cite{MVC:cakephp}.

\subsection{Focus Areas}
I will focus on performance and will also fake a high traffic to see how different platforms will handle it. This is a important test to see how a framework can be used in different business and in different websites sizes.
Performance can be different things. But we will focus mostly on page loads.

This focus areas is important to see how a framework can be used in different business and in different websites sizes.

\subsection{Goals}
The goals is to know which framework is good in different ways. Like high traffic, speed while loading a page etc. This means that I will, according to my data choose which one is better in different situations, such as big Company, Medium Company and Small/individual.
\section{Research}
\subsection{Research Questions}
\begin{description}
  \item[RQ1:] Which framework can handle traffic fastest?
  \item[RQ2:] What is the weakness in each framework in performance?
  \item[RQ3:] which framework is best suited in the different sizes of a company (big, medium and small/individual)?
\end{description}


\subsection{Research Methodology}
  \begin{tabular}{ l  r }
     
    Research Question & Methodology \\ 
    \hline
    RQ1 & Technical Experiment \\ 
    RQ2 & Technical Experiment / Literature\\
    RQ3 & Technical Experiment \\
  \end{tabular}
  
\subsubsection{Collecting Data}
To answer RQ1 and to RQ3: The high traffic load will be testing to a computer on the cloud (Amazon) with same specifications for both CakePHP and Codeigniter. I will be using Openload that will simulate high traffic through http. The server will be running default set Apache with PHP 5.*, probably 5.3.10, if the new 5.4 isn't reliable.
I run high traffic in 3-5 phases 15 times, depending on the sizes of the company on each framework. making it total of 45 tests that will be run randomly for each framework to then take the average of each sizes.
this can of course be changed in the real thesis if i find better High traffic simulator or better solution to do the tests.
Each framework will have a simple guestbook application with database that will get different sizes of high traffic depending on sizes of the website, no difference on server for the difference sizes of the website.

To answer RQ2: Collecting literature and doing some own data collecting for the differences between the framework on their sites. The literature will be about the good of one framework or talking about some differences in MVC frameworks, as example. See the literature review design.
  
\subsubsection{Literature review design}
Criteria:
\begin{itemize}
  \item Should have topic similar to php framework
  \item Should have something about performance or difference of framework
  \item Should have good context of data to reference to
  \item Have something to tell why one or another framework is better or bad or differences between.
\end{itemize}

Search terms:
\begin{enumerate}
  \item PHP performance
  \item PHP framework
  \item Codeigniter performance
  \item CakePHP performance
  \item CakePHP Codeigniter
\end{enumerate}
This search terms is in google-thinking search terms, will maybe be different from each place if they are more strict to get the best results.

I will search on Web, Google Scholar, Elin, Engineering village and IEEE, but also on library. because of the difference of amount of literatures to get of the search terms above. It will not be specific criteria for each of them. I will take them with a good point according the unique framework is in the literature and how it can be in good support for my work, such as difference in php frameworks, how both frameworks of this thesis work and if any, have something about high traffic on each of the framework.
%
% The following two commands are all you need in the
% initial runs of your .tex file to
% produce the bibliography for the citations in your paper.
\bibliographystyle{abbrv}
\bibliography{sigproc}  % sigproc.bib is the name of the Bibliography in this case
% You must have a proper ".bib" file
%  and remember to run:
% latex bibtex latex latex
% to resolve all references
%
% ACM needs 'a single self-contained file'!
%
%APPENDICES are optional
%\balancecolumns


% That's all folks!
\end{document}
